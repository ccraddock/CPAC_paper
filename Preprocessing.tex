\section{Preprocessing}
\textit{This is the section that describes preprocessing}

\subsection{Motion Correction}
\text{Movement during scanning is one of of the largest factors influencing the quality of fMRI data. Movement of the head between the acquisition of each volume can cause brain images to become misaligned. Head motion during scanning can also cause in spurious changes in signal intensity. If they are not corrected, these changes can influence the results of activation and connectivity analyses. Recent studies have shown that motion as small as 0.1mm can systematically bias both within- and between- group effects during the analysis of fMRI data (Power et al., 2011; Satterhwaite et al., 2012; Van Dijk et al., 2012). Even the most cooperative subjects often still show displacements of up to a millimeter, and head movements of several millimeters are not uncommon in studies of hyperkinetic subjects such as young children, older adults, or patient populations.

There are three main approaches to motion correction: volume realignment, using a general linear model to regress out motion-related artifacts (i.e. regression of motion parameters), and censoring of motion confounded time points (i.e. “scrubbing”)}

\subsubsection{Volume Realignment}
\text{Volume realignment aligns reconstructed volumes by calculating motion parameters based on a solid-body model of the head and brain (Friston 1996). Based on these pareameters, each volume is registered to the volume preceeding it.

CPAC runs volume realignment on all functional images during functional preprocessing. Users can select from two sets of motion parameters to use during this process:}

\begin{itemize}
    \item{\textbf{6-Parameter Model} - Three translation and three rotation parameters as described in Friston 1996.}
    \item{\textbf{Friston 24-Parameter Model} - The 6 motion parameters of the current volume and the preceeding volume, plus each of these values squared.}
\end{itemize}



\begin{itemize}
    \item Anatomical Preprocessing
    \begin{itemize}
        \item Basic preprocessing
        \item Segmentation
        \item Non-linear registration to template space
    \end{itemize}
    \item Functional Preprocessing
    \begin{itemize}
        \item Basic preprocessing, slice timing correction, and motion correction
        \item Coregister mean functional to the anatomical
        \item Denoising strategies
        \begin{itemize}
            \item Physiological noise correction
            \item White matter, CSF
            \item Global signal regression
            \item motion correction
        \end{itemize}
    \end{itemize}
\end{itemize}
